% !TEX program = xelatex

\documentclass{resume}
\usepackage{setspace}
\usepackage{zh_CN-Adobefonts_external} % Simplified Chinese Support using external fonts (./fonts/zh_CN-Adobe/)
% \usepackage{zh_CN-Adobefonts_internal} % Simplified Chinese Support using system fonts

\begin{document}
\pagenumbering{gobble} % suppress displaying page number

\name{单子奇}

\basicInfo{
  \email{breeze.shane@qq.com} \textperiodcentered\ 
  \phone{(+86) 138-0423-2322} \textperiodcentered\ 
  \faGroup{意向岗位: C++/Rust 开发}
}
\centerline{
  \github[BreezeShane: https://github.com/BreezeShane]{https://github.com/BreezeShane}
  \textperiodcentered\ 
  \faBook{\href{http://breezeshane.github.io/}{技术博客: http://breezeshane.github.io/}}
}


\section{\faGraduationCap\  教育背景}
\datedsubsection{\textbf{桂林电子科技大学}, 桂林, \textit{本科在读}\ 软件工程专业}{2020年 -- 2024年}
\textit{学分绩}: \underline{  86.7/100  }\quad
\textit{本专业排名}: \underline{  5/150  }\quad
\textit{外语}: 大学英语4级(CET 4): \underline{  459.0  }

\section{\faUsers\ 项目经历}
\datedsubsection{\textbf{基于C++\&Qt的音乐播放器}}{2023 年6月 -- 2023年7月}
\role{C++, Qt}{合作项目, https://github.com/BreezeShane/PineappleMusic}
\begin{onehalfspacing}
基于C++和Qt实现的音乐播放器, 本人完成其中歌词滚动与播放控制的功能。
\begin{itemize}
  \item \textit{项目设计存在缺陷}: 列表主页面和歌词详情页面被设计成两个窗口。这个缺陷导致了两个页面的媒体控制存在冲突。
  \item \textit{解决方案有两种}: 
  \begin{itemize}
    \item 将歌词详情页面以折叠窗口(作为控件组)的形式放入列表主页面;
    \item 两个窗口共享一个媒体流控制进程,两窗口上的媒体控制按钮向该进程发送信号,而媒体流控制进程专门负责媒体控制功能。
  \end{itemize}
\end{itemize}
\end{onehalfspacing}

\datedsubsection{\textbf{基于生成对抗网络的低照度增强系统}}{ 2021 年 8 月 -- 2022 年 10 月}
\role{Python}{个人项目, https://github.com/BreezeShane/NovelEnlightenGAN}
\begin{onehalfspacing}
参考论文实现完成的低照度增强系统
\begin{itemize}
  % \item 对原有项目进行精简化,去除冗余的代码
  % \item 基于Flask框架完成前后端交互
  % \item 对系统配置进行集中化处理,便于管理系统
  \item 深入理解了从公式到项目的具体转化过程
  \item 进一步理解了生成对抗网络的结构、训练方式及损失函数的设计
  \item 熟悉了从实验项目到生产项目、从实验设计到落实应用的基本转化
  \item 熟悉了模块化设计理念和面向对象设计理念
\end{itemize}
\end{onehalfspacing}

\datedsubsection{\textbf{微型图书管理系统}}{2022 年5月 -- 2022年6月}
\role{Shell, Python}{个人项目, https://github.com/Tiny-LIMS-Team/LibraryInfoManager}
\begin{onehalfspacing}
面向中小型的图书管理系统, 本项目从设计到开发均由个人完成
\begin{itemize}
  % \item 基于Django框架开发的Web应用程序
  % \item 使用MySQL数据库支持运行, 并通过 ORM 方式管理数据库
  \item 了解Django前后端的交互原理及设计
  \item 熟悉Django和常见数据库的链接配置, 了解Django中数据库的 ORM 管理方式
  \item 熟悉数据库表的设计和数据库多表关系的设计
  \item 了解功能设计与实现和业务逻辑之间的相互转化
\end{itemize}
\end{onehalfspacing}

\datedsubsection{\textbf{云存储服务系统}}{2022年12月 -- 2023年1月}
\role{Vue, Python}{合作项目, https://github.com/Tiny-LIMS-Team/HadoopNetDisk}
\begin{onehalfspacing}
多用户云存储服务系统后端开发, 本人完成了整个项目的后端设计与业务逻辑功能,并参与实现了后端与集群的交互
\begin{itemize}
  % \item 采用了JwT安全认证和C-P-S模式即客户端-代理-服务端进行服务支持
  % \item 基于Apache Hdfs文件系统、MySQL与HBase数据库完成开发
  \item 熟悉JwT安全认证的用户权限校验方式, 并首次在本项目中设计与实现用户权限校验的功能
  \item 熟悉 Docker Compose 与环境搭建之间的关系, 能使用 Docker Compose 文件创建完整环境生态。
  \item 基本熟悉Apache Hdfs文件系统的搭建(包括环境配置)和应用, 并基本熟悉HBase数据库的使用
  \item 了解关系型数据库MySQL和非关系型数据库HBase之间的联系与区别
  \item 了解后端设计上的C-P-S模式即客户端-代理-服务端模式
\end{itemize}
\end{onehalfspacing}

\section{\faHeartO\ 获奖情况}
\datedline{\textit{企业命题类第三名}, 第十二届中国大学生服务外包创新创业大赛}{2021 年8 月}
\datedline{\textit{铜奖}, 第七届中国国际“互联网+”大学生创新创业大赛“数广集团杯”广西赛区}{2021 年8 月}
\datedline{\textit{2020-2021学年度}, 国家励志奖学金}{2021 年 12 月}
\datedline{\textit{2021-2022学年度}, 国家励志奖学金}{2022 年 12 月}

\section{\faCogs\ IT 技能}
% increase linespacing [parsep=0.5ex]
\begin{itemize}[parsep=0.5ex]
  \item 编程语言: Python > Rust $\approx$ C > C++ > Java > Lua
  \item 平台: Linux
  \item 框架: Django, PyTorch
\end{itemize}

% \section{\faInfo\ 其他}
% % increase linespacing [parsep=0.5ex]
% \begin{itemize}[parsep=0.5ex]
%   \item 技术博客: http://breezeshane.github.io/
%   \item GitHub: https://github.com/BreezeShane
% \end{itemize}

%% Reference
%\newpage
%\bibliographystyle{IEEETran}
%\bibliography{mycite}
\end{document}
