% !TEX program = xelatex

\documentclass{resume}
\usepackage{setspace}
\usepackage{zh_CN-Adobefonts_external} % Simplified Chinese Support using external fonts (./fonts/zh_CN-Adobe/)
% \usepackage{zh_CN-Adobefonts_internal} % Simplified Chinese Support using system fonts

\begin{document}
\pagenumbering{gobble} % suppress displaying page number

\name{单子奇}

\basicInfo{
  \email{breeze.shane@qq.com} \textperiodcentered\ 
  \phone{(+86) 138-0423-2322} \textperiodcentered\ 
  \faGroup{意向岗位: C++/Rust/Python 研发}
}
\centerline{
  \github[BreezeShane: https://github.com/BreezeShane]{https://github.com/BreezeShane}
  \textperiodcentered\ 
  \faBook{\href{http://breezeshane.github.io/}{技术博客: http://breezeshane.github.io/}}
}


\section{\faGraduationCap\  教育背景}
\datedsubsection{\textbf{桂林电子科技大学}, 桂林, \textit{本科在读}\ 软件工程专业}{2020年 -- 2024年}
\textit{学分绩}: \underline{  86.7/100  }\quad
\textit{本专业排名}: \underline{  5/150  }\quad
\textit{外语}: 大学英语4级(CET 4): \underline{  459.0  }

\section{\faUsers\ 项目经历}

\datedsubsection{\textbf{昆虫识别系统}}{ 2024 年 2 月 -- 2024 年 6 月}
Rust, React.js - 个人项目, https://github.com/BreezeShane/GraduationDesign \newline
\begin{onehalfspacing}
基于 Rust 和 React.js 实现的昆虫识别系统, 包括前后端部分和深度学习部分均由个人设计开发完成
  % \begin{itemize}
%   \item \textit{项目设计存在缺陷}: 列表主页面和歌词详情页面被设计成两个窗口。这个缺陷导致了两个页面的媒体控制存在冲突。
%   \item \textit{解决方案有两种}: 
%   \begin{itemize}
%     \item 将歌词详情页面以折叠窗口(作为控件组)的形式放入列表主页面;
%     \item 两个窗口共享一个媒体流控制进程,两窗口上的媒体控制按钮向该进程发送信号,而媒体流控制进程专门负责媒体控制功能。
%   \end{itemize}
% \end{itemize}
\end{onehalfspacing}

\datedsubsection{\textbf{菠萝音乐播放器}}{2023 年6月 -- 2023年7月}
C++, Qt - 合作项目, https://github.com/BreezeShane/PineappleMusic \newline
\begin{onehalfspacing}
基于C++和Qt实现的音乐播放器, 本人完成其中歌词滚动与播放控制的功能
% \begin{itemize}
%   \item \textit{项目设计存在缺陷}: 列表主页面和歌词详情页面被设计成两个窗口。这个缺陷导致了两个页面的媒体控制存在冲突。
%   \item \textit{解决方案有两种}: 
%   \begin{itemize}
%     \item 将歌词详情页面以折叠窗口(作为控件组)的形式放入列表主页面;
%     \item 两个窗口共享一个媒体流控制进程,两窗口上的媒体控制按钮向该进程发送信号,而媒体流控制进程专门负责媒体控制功能。
%   \end{itemize}
% \end{itemize}
\end{onehalfspacing}

\datedsubsection{\textbf{基于生成对抗网络的低照度增强系统}}{ 2021 年 8 月 -- 2022 年 10 月}
Python - 个人项目, https://github.com/BreezeShane/NovelEnlightenGAN \newline
\begin{onehalfspacing}
参考论文实现完成的低照度增强系统
% \begin{itemize}
%   % \item 对原有项目进行精简化,去除冗余的代码
%   % \item 基于Flask框架完成前后端交互
%   % \item 对系统配置进行集中化处理,便于管理系统
%   \item 深入理解了从公式到项目的具体转化过程
%   \item 进一步理解了生成对抗网络的结构、训练方式及损失函数的设计
%   \item 熟悉了从实验项目到生产项目、从实验设计到落实应用的基本转化
%   \item 熟悉了模块化设计理念和面向对象设计理念
% \end{itemize}
\end{onehalfspacing}

\datedsubsection{\textbf{微型图书管理系统}}{2022 年5月 -- 2022年6月}
Shell, Python - 个人项目, https://github.com/Tiny-LIMS-Team/LibraryInfoManager \newline
\begin{onehalfspacing}
面向中小型的图书管理系统, 本项目从设计到开发均由个人完成
% \begin{itemize}
%   % \item 基于Django框架开发的Web应用程序
%   % \item 使用MySQL数据库支持运行, 并通过 ORM 方式管理数据库
%   \item 了解Django前后端的交互原理及设计
%   \item 熟悉Django和常见数据库的链接配置, 了解Django中数据库的 ORM 管理方式
%   \item 熟悉数据库表的设计和数据库多表关系的设计
%   \item 了解功能设计与实现和业务逻辑之间的相互转化
% \end{itemize}
\end{onehalfspacing}

\datedsubsection{\textbf{云存储服务系统}}{2022年12月 -- 2023年1月}
Vue, Python - 合作项目, https://github.com/Tiny-LIMS-Team/HadoopNetDisk \newline
\begin{onehalfspacing}
多用户云存储服务系统后端开发, 本人完成了整个项目的后端设计与业务逻辑功能,并参与实现了后端与集群的交互
% \begin{itemize}
%   % \item 采用了JwT安全认证和C-P-S模式即客户端-代理-服务端进行服务支持
%   % \item 基于Apache Hdfs文件系统、MySQL与HBase数据库完成开发
%   \item 熟悉JwT安全认证的用户权限校验方式, 并首次在本项目中设计与实现用户权限校验的功能
%   \item 熟悉 Docker Compose 与环境搭建之间的关系, 能使用 Docker Compose 文件创建完整环境生态。
%   \item 基本熟悉Apache Hdfs文件系统的搭建(包括环境配置)和应用, 并基本熟悉HBase数据库的使用
%   \item 了解关系型数据库MySQL和非关系型数据库HBase之间的联系与区别
%   \item 了解后端设计上的C-P-S模式即客户端-代理-服务端模式
% \end{itemize}
\end{onehalfspacing}

\section{\faCogs\ IT 技能}
% increase linespacing [parsep=0.5ex]
\textbf{编程语言}: 熟悉 Python, 掌握 Rust 和 C/C++, 掌握 React.js, 了解 Vue.js, Java, Lua, Shell \newline
\textbf{平台}: 熟悉 Arch系的 Linux 的操作, 了解 Linux 的一些运作原理 \newline
\textbf{框架}: 熟悉 Django 前后端应用框架, 熟悉 PyTorch 深度学习框架 \newline
\textbf{工具}: 掌握 Docker 工具的使用 \newline
\textbf{数据库}: 熟悉 MySQL 关系型数据的使用, 了解 MongoDB, HBase 等非关系型数据库 \newline
\textbf{掌握的项目技能}: 
\begin{itemize}[parsep=0.5ex]
  \item 熟悉从实验项目到生产项目、从实验设计到落实应用的基本转化, 熟悉生成对抗网络的结构、训练方式及损失函数的设计
  \item 熟悉功能设计与实现和业务逻辑之间的相互转化
  \item 熟悉模块化设计理念和面向对象设计理念, 熟悉数据库表的设计和数据库多表关系的设计
  \item 熟悉 Django 和常见数据库的链接配置, 了解 Django 中数据库的 ORM 管理方式的基本原理
  \item 熟悉 JwT 安全认证的用户权限校验方式, 并完成过用户权限校验功能的设计与实现
  \item 熟悉 Docker Compose 与环境搭建之间的关系, 能使用 Docker Compose 文件创建完整环境生态。
  \item 掌握 Apache Hdfs 文件系统的搭建(包括环境配置)和应用
  \item 掌握 HBase 数据库的基本使用
  \item 了解 Django 前后端的交互原理及设计
  \item 了解关系型数据库 MySQL 和非关系型数据库 HBase 之间的联系与区别
  \item 了解后端设计上的 C-P-S 模式即客户端-代理-服务端模式
\end{itemize}

\section{\faHeartO\ 获奖情况}
\datedline{企业命题类第三名, 第十二届中国大学生服务外包创新创业大赛}{2021 年8 月}
\datedline{铜奖, 第七届中国国际“互联网+”大学生创新创业大赛“数广集团杯”广西赛区}{2021 年8 月}
\datedline{2020-2021学年度, 国家励志奖学金}{2021 年 12 月}
\datedline{2021-2022学年度, 国家励志奖学金}{2022 年 12 月}

% \section{\faInfo\ 其他}
% % increase linespacing [parsep=0.5ex]
% \begin{itemize}[parsep=0.5ex]
%   \item 技术博客: http://breezeshane.github.io/
%   \item GitHub: https://github.com/BreezeShane
% \end{itemize}

%% Reference
%\newpage
%\bibliographystyle{IEEETran}
%\bibliography{mycite}
\end{document}
